\begin{abstract}
    Air quality is an important aspect of a room.
    It affects the health, comfort, and
    productivity of the occupants. This paper
    presents an Internet-of-Things based system
    that can be used to monitor the air quality of
    a room and the outside air, and control the
    door/window of the room in hope of achieving a
    better air quality. The system will be able to
    monitor the air quality of a room, with
    temperature and humidity as the indicators,
    using DHT22 sensor; and the outside, with
    carbon dioxide (CO$_2$) ppm as the indicator,
    using MQ135 sensor; and ESP32 as the
    microprocessor. The system also utilizes
    Time Series KMeans (TSKM) to determine the
    quality of the indicators, to decide whether
    to open or close the door.
\end{abstract}

\begin{IEEEkeywords}
    Air quality, Internet-of-Things, ESP32, DHT22, MQ135, Time Series KMeans, Web Application
\end{IEEEkeywords}
