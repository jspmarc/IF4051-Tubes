\section{Literature Review}

This section presents a literature review and definitions of key concepts and
components that is used by the proposed system.

\subsection{Air Quality}

Air quality refers to how healthy and comfortable the air is. In this paper, the
quality level is determined by:
\begin{itemize}
	\item the parts per million (ppm) of CO$_2$,
	\item temperature, and
	\item humidity level (in percent)
\end{itemize}
of the air.

In 2020, the Water and Air Quality Bureau of Health Canada proposed a long-term
exposure limit for CO$_2$ should not exceed 1000 ppm (based on 24 hours
average). Exposure to high CO$_2$ ppm can lead to health issues, such as dry
eyes and tight chest, and also neurological effects that affect cognitive
perofrmance. Some studies have shown that decrease in CO$_2$ ppm leads to better
better task performance and decrease in sick building syndrome
symptoms\cite{co2_ppm_max}.

The comfortable temperature and humidity in this paper is determined by the
average temperature range and average humidity of Bandung City, West Java,
Indonesia. According to weather-and-climate.com, the temperature in Bandung
usually ranges from 22\textdegree C to 30\textdegree C\cite{temperature_max}.
For humidity, the same source is used but the city is in Bogor, West Java
because there's no humidity data in Bandung and Bogor has geographic similarity
to Bandung which includes the humidity. According to said source, Bogor has an
average humidity of 80\%\cite{humidity_max}.

%% JOSEP: MIGHT NOT FIT FOR CHAPTER 2
% Based on the proposed maximum CO$_2$ ppm by Water and Air Quality Bureau of
% Health Canada, to have a good air quality, the CO$_2$ ppm has to be as low as
% possible. It is also observed to have a good air quality, the humidity and
% temperature should be a ``perfect'' value which is not be too high.

\subsection{K-means Clustering}

K-means clustering is one of the algorithms that can be used for
unsupervised learning in machine learning. With unsupervised learning,
the engineer doesn't have to label each data sample, instead the machine will
learn from the data and apply label in its own. This is very helpful when
the given dataset is unlabeled and comes in high quantity.

K-means clustering is also known as squared-error clustering because the
objective is to obtain a partition which, for a fixed number of clusters,
minimizes the square-error \cite*{clustering_fariska}. Square-error is the sum
of the Euclidean distance between each data sample in a cluster and its cluster
center.

In k-means clustering, clusters will be created and using the means of each
samples in the cluster. Then, the machine learning algorithm will label a data
sample with a cluster where the mean of said cluster is the to the sample's
data.

The alogrithm of training an ML model using k-means clustering to create a model
with $k$ clusters is as follows\cite{clustering_fariska}:
\begin{enumerate}
	\item randomly choose $k$ data points from the dataset $D$,
	\item repeat until convergence (no change in cluster):
	      \begin{enumerate}
		      \item assign (or reassign if needed) each data point to a cluster
		            where it is the most similar (based on the value).
		      \item update the cluster means.
	      \end{enumerate}
\end{enumerate}

\subsection{ESP32}

ESP32 is a family of chips with low-cost and low-power SoC (system on a chip)
with Wi-Fi and Bluetooth by Espressif systems \cite{esp32_net}. ESP32 has a single
32-bit processor with up 2 cores and clock speed of up to 240 MHz. This results
in a performance of up to 600 DMIPS. Additionally, all ESP32 models have 448 KiB
of ROM and 520 KiB of SRAM\cite{esp32_datasheet}.

ESP32 systems can be connected wirelessly using Bluetooth or Wi-Fi. For Wi-Fi,
ESP32 fully supports TCP/IP and 802.11 b/g/n Wi-Fi MAC protocol. While on the
Bluetooth side, ESP32 supports Bluetooth V4.2 BR/EDR and Bluetooth LE (low energy).
ESP32 can also be connected using USB through serial port\cite{esp32_datasheet}.

Programming and flashing ESP32 can be done using ESP-IDF (Espressif IoT
Development Framework). ESP-IDF the official IoT development framework for ESP32
created by Espressif Systems. Although other frameworks and IDE, such as
platform.io and Arduino IDE, also supports creating and loading programs for and
to ESP32.

\subsection{Message Queuing Telemetry Transport Broker (MQTT)}

MQTT is an OASIS standard messaging protocol that's usually used for IoT
development. It is also known as ISO/IEC 20922:2016. MQTT is designed to be
light-weight, open, simple, and easy to implement\cite{mqtt_iso}. MQTT runs over
TCP/IP or other networking protocol which provides the same characteristics as
TCP/IP (e.g. TLS).

MQTT has the following features:
\begin{itemize}
	\item uses pub/sub messaging pattern which provides one-to-many message
	      distribution,
	\item a messaging transport that's agnotic to the content payload,
	\item has the three qualities of message delivery: at-most-once,
	      at-least-once, exactly-once, and
	\item uses broker and topics.
\end{itemize}
The features and characteristics of MQTT makes it suitable for IoT usage
where the embedded computer is usually low-powered and might have unreliable
network.

MQTT also has some additional security features, such as the use of TLS
for transport protocol. MQTT also supports modern authentication protocols, such
as OAuth\cite{mqtt_org}.

\subsection{Apache Kafka}

Apache Kafka is an open-source message queue system by Apache Software
Foundation. Its usage is similar to MQTT where it has a broker and topics and
the clients send messages in a pub/sub manner. The difference between Kafka and
MQTT lies in the implementation. MQTT is a standard, Kafka is already
the implementation. Kafka brokers store message logs which means client can
consume it later (i.e. not realtime) as long as the message log still exists.
Also, Kafka can have multiple brokers, whereas MQTT really depends on the
implementation.

\subsection{Redis}

Redis is a NoSql database that uses a key-value schema. Redis saves its data
in-memory\footnote{Redis has Redis Persistance to periodically save data to
	secondary memory as backup}, which means it's perfect to be used for
caching. Other than for caching, Redis can also used to save state between
application. Because it has a fast I/O (when ignoring network), multiple
instances of an application can connect to the same Redis instance and share
their state through Redis.

\subsection{InfluxDB and Telegraf}

InfluxDB is a NoSql database with high read and write performance. Data can be
written and read in realtime. It is also included ETL (extract, transform, and
load), monitoring, user dashboard, and data visualization tools out of the box.
The features of InfluxDB makes it a perfect DB for saving IoT sensor data which
is captured at a fast rate. In InfluxDB, tables are similar to measurements and
databases are similar to buckets.

InfluxDB also has features for security. It includes authentication where user
needs to login before they can access the dashboard. It also has authorization
to access its buckets in the form of API token.

To insert data to InfluxDB buckets, its user can use the InfluxDB API, SDK, or
Telegraf. Telegraf is a server agent to collect data from systems and sensors.
Telegraf can be attached to MQTT to retrieve data from MQTT and then save them
directly to InfluxDB.

\subsection{Apache Spark}

Apache Spark is a product of Apache Software Foundation. Apache Spark is mostly
used for big data processing. Spark is usually used for ETL processes for big
data. Apache Spark can also be used to train machine learning models in real
time.

Apache Spark is able to process large amount of data in relatively short time
because it uses primary memory to save temporary data instead of writing them to
a file. Apache Spark can also be distributed accross machines spread all over
the world or be run in a multiprocess configuration in a single machine.