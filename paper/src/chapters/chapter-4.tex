\section{Result and Discussion}
This section presents the results of the system,
and discusses the results.

\subsection{Result}
The results of the system can be seen in the web
application. The system can successfully gather the data from the
MQ135 and DHT22 sensors, and can display the data in
the web application, as seen in figure \ref{webapp-stats-view}.
There are a few time intervals to choose from to display the data.
The system can successfully open and close the door by
determining the current average of the indicators,
and act accordingly. As seen in figure \ref{webapp-home-view},
the web application is an user interface to control the system,
to change system mode, and to open or close the door remotely.
At the time of writing this paper, the alarm feature is not yet functional.
Figure \ref{webapp-alerts-view} shows the alerts view of the web application.
The alerts shown are alerts that sent within the last 12 hours.
A token must be inputted in the web application to be able to control the system,
to view statistics, and to view alerts.

\begin{figure}
      \centerline{\includegraphics[scale=0.2]{resources/webapp-home-view.png}}
      \caption{Home view of the web application of the smart room air conditioner}
      \label{webapp-home-view}
\end{figure}

\begin{figure}
      \centerline{\includegraphics[scale=0.4]{resources/webapp-stats-view.png}}
      \caption{Statistics view of the web application of the smart room air conditioner}
      \label{webapp-stats-view}
\end{figure}

\begin{figure}
      \centerline{\includegraphics[scale=0.5]{resources/webapp-alerts-view.png}}
      \caption{Alerts view of the web application of the smart room air conditioner}
      \label{webapp-alerts-view}
\end{figure}

\subsection{Discussion}
\subsubsection{Sensors}
The sensors used in the system could detect the
indicators of the air quality, but the accuracy
of the sensors are still questionable.
Specifically, the MQ135 sensor, which is used to
detect the carbon dioxide (CO$_2$) ppm, is not
accurate enough to be used as a reference for.
The MQ135 sensor's base resistance (RZERO) is
not stable, thus affecting the sensor's
reading of the CO$_2$ ppm.

\subsubsection{Machine Learning}
Although the system could already determine to open
or close the door, the system is still far from
perfect. The decision making of opening and closing
the door is merely based on the data from the
sensors and the clustering model. While the actual
air quality of the room related to the health,
comfort, and productivity of the inhabitant are not
considered.

\subsubsection{System Architecture}
Other issues is that, architeturally, the machine
learning model is deployed in the same server with
the application back-end, thus lowering the
performance of the application back-end, when the
machine learning model is being used.
