\section{Conclusion and Suggestions}

\subsection{Conclusion}
In this paper, we have presented a system that can be
used to monitor and control the air quality of a room.
The system is able to monitor the air quality of a room,
with temperature and humidity as the indicators; and
the outside, with carbon dioxide (CO$_2$) ppm as the indicator;
and uses those informations to make decision about when to
open and close doors/windows.
The system is also able to be controlled manually via a web
application, and sends notifications to the user's phone
about the state of the system.

\subsection{Suggestions}
The system we proposed is still far from perfect.
There are still many things that can be improved, such as:
\begin{enumerate}
      \item More sophisticated and accurate sensors are
            used.
      \item Health, comfort, and productivity related
            aspects of air quality are considered,
            and not merely depend on the data and
            clustering algorithm.
      \item The machine learning model is deployed
            in a separate server, so that the
            application back-end is not affected
            by the machine learning model.
\end{enumerate}
